\documentclass[titlepage]{article}

\author{Brad Stewart and Danny DiPaolo}
\title{Moobot User Manual}
\date{\today}

\begin{document}
	\maketitle
	\tableofcontents
	\newpage

	\section{Meta}
		\subsection{Who Should Read This?}
			This document is intended primarily for anyone who is
			running/admining a moobot, though the information herein will also
			be of use to anyone who frequents a channel inhabited by a moobot.
		\subsection{What Is Covered?}
			My intent in writing this is to give people not familiar with what
			moobot is and how it works a heads up on how to run, use and
			administer it.  The included moduels, permission system, and
			module interface are discussed.
		\subsection{What's Not covered?}
			This isn't intended to be an installation/setup howto, for that
			check the installation manual, which should be available in the
			same place you got this document.
		\subsection{Latest Version}
			The latest version of this document can be retrieved from the
			moobot project site at http://moobot.sourceforge.net.
	\section{Introduction}
		\subsection{Who We Are}
			The two primary developers on this project, Brad Stewart and Danny
			DiPaolo, are a college student and a recent college graduate
			(respectively) who are Python fanatics and IRC fans. 

			Though we are responsible for the core development, others help by
			writing modules, reporting bugs, and all that good stuff.  If you
			ever want to chat, one or both of us can usually be found on
			irc.oftc.net in \#moobot or \#grasshoppers.  We go by Bradmont and
			Strike, respectively.
		\subsection{Why We Are Doing This}
			As I've already mentioned, we're IRC fans.  At one point, the
			previous host of the IRC bot that lived in a channel we both
			frequent (a blootbot) took down his server, and the bot with it.
			At this point, Brad stepped in, downloaded the latest blootbot
			release, and suffered through its install and configuration.  
			
			Running this bot was a bit of a hastle, with semi-frequent
			breakages, functions that didn't work, and all sorts of other
			joys.  Because it's written in perl, debugging any errors he found
			was no easy task.  Anyway, to make a long story even longer, one
			day we got to talking about the problems with the blootbot, and
			thought it might be fun to create our own.  We registered a
			project on sourceforge, and then quite happily let it sit,
			untouched, for five months.

			So eventually, during one Christmas vacation, we got back to
			talking about the moobot, and did a bit of playing.  We stumbled
			upon a nice library, irclib, written by Joel Rosdahl.  This
			provided all the networking/protocol code for us, as well as a
			basic ircbot class, from which we built the initial revision of
			moobot.  Though the current moobot uses this library set as its
			base, we have plans to redo the whole foundation of the bot, but
			that's going to take awhile. (Ed. note from Strike:  the more I
			look at it, the more I realize irclib is pretty good as is, but we
			shouldn't rely on the ircbot class)

			Anyway, after about 3 days straight hacking, we had a reasonably
			usable bot, which has evolved since then into what it is today.
			Enjoy. :-)
		\subsection{WTF is moobot?}
			You've probably inferred this from that intro, but it's an IRC bot
			written in Python (the language of the gods, thank you very much
			Guido).  It's intended to be similar to blootbot (from a user's
			standpoint), but much more uniform, normal, and manageable.

	\section{Privileges/Grants System}
		Because I'd be unable to write the rest of this document without
		constantly referring to the privs systm, I'll cover that first.

		Moobot has a (reasonably) decent privilege system, to control
		everything from joining/parting channels to ignoring troublesome
		users.  A privilege is granted to a nick/hostmask (eg,
		Bradmont!bradmont@*.cc.shawcable.net) and anyone who matches that
		hostmask gets the priv.  For this reason, be very selective about what
		hostmasks you give a privilege to.

		The current privileges are:
		\begin{itemize}
			\item[all\_priv]  This counts as every privilege (except
			notalk\_priv).  You likely only want to give this priv to
			yourself, and possibly others who you absolutely trust to have
			full power over the bot.
			\item[grant\_priv]  grant\_priv allows a person to grant other
			privileges they have (including grant\_priv) to anyone else.  You
			will also want to be careful with this one.
			\item[delete\_priv]  delete\_priv gives a user the ability to
			delete any factoid in the database, whether they created it or
			not.
			\item[lock\_priv] This priv lets a user lock or unlock any factoid
			in the database (normally you can only lock factoids that match
			your nick, and unlock ones you've locked)
			\item[notalk\_priv] notalk\_priv causes the bot to totally ignore
			everything a given user says.
			\item[join\_priv]  Users with join\_priv can tell the bot to join
			channels.
			\item[part\_priv]  Users with part\_priv can tell the bot to part
			channels.
			\item[nick\_priv]  This priv lets the grantee tell the bot to
			change it's nick.
			\item[kick\_priv]  kick\_priv gives a user the power to have the
			bot kick someone from a channel (provided the bot is opped in that
			channel, of course).
			\item[add\_lart\_priv] With this priv, a user can add larts,
			praises and dunnos to the database.
			\item[poll\_priv]  This priv allows users to create, alter and
			delete polls.
			\item[reset\_stats\_priv]  With this priv, a user can reset a
			user's stats (eg, hehstats, lolstats, karma).
			\item[faketell\_priv] Normally, when you get the bot to tell
			(user) about (factoid), the bot will do so, and send the message
			in a format like ``Bradmont wanted you to know:  foo is bar.''
			with this priv, you can use the alternate ``faketell,'' which will
			remove the ``Bradmont wanted you to know:'' from the beginning of
			the message, making it look sort of like the bot is talking of its
			own accord.
		\end{itemize}

	\section{IRC Functions}
		What follows is a fairly comprehensive description of a few of the
		more popular functions included with Moobot.  It is in no way intended
		to be comprehensive of all modules, as that number continues to grow.
		Eventually we hope to implement an online help system that makes the
		modules self-documenting and reduces the amount of documentation we
		have to write.
	
		\subsection{Join/Part}
			To have the bot join or part a channel, give it the command ``join
			\#channel'', eg, ``moobot: join \#grasshoppers''.  This can be
			done in public or via a privmsg directly to the bot.  Parting
			works exactly the same, ``moobot: part \#channel''.  To do these
			two actions, you need join\_priv and part\_priv, respectively (or
			all\_priv, of course).
		\subsection{Kicking}
			Ever want to kick someone but you didn't have ops?  Well, now you
			can!  You just have to have your bot with ops in that channel.
			Simply tell your bot ``kick \#channel nick'' and it will kick the
			user specified by ``nick'' out of \#channel.  You need kick\_priv
			to do this.

	\section{Factoids}
		Factoids are one of the main reasons we wanted a bot in our channel.
		It's easy to store useful (or not so useful) information for ALL to
		read by simply telling the bot about it and then either having the bot
		tell them later or letting them discover it in whatever way they
		choose.
		
		\subsection{Requesting Factoids}
			If you know the name of the factoid of that you wish to see, then
			simply address the bot with the factoid key and it will return the
			value for you.  Some factoids contain special groups which get
			chosen randomly, such as ``(foo|bar)'', which will get parsed to
			choose either ``foo'' or ``bar'' when a factoid is requested.  To
			see the actual factoid value without any of this parsing taking
			place, you simply address the bot with ``literal <factoid name>''.
	
		\subsection{Adding Factoids}
			Factoids are added by simply telling the bot ``foo is bar'' (of
			course, addressing it either explicitly with the bot name or with
			the shorthand addressing style).  Of course, a number of things
			can happen as a result of this.  The factoid may already exist,
			which the bot will then tell you.  Otherwise, the factoid will be
			added to the bot's factoid database.  By default, anyone can add
			factoids, but if they are ignored, they cannot.

			Note: as naming convention goes, ``foo'' is called the ``factoid
			key'' and ``bar'' is the ``factoid value''.

		\subsection{Deleting Factoids}
			Factoids are removed by simply telling the bot to either ``forget''
			or ``delete'' the factoid key.  If the person has delete\_priv and
			the factoid is not locked, it will be removed.

		\subsection{Locking and Unlocking Factoids}
			To lock a factoid the person who requests it must either have
			lock\_priv or the factoid key they are trying to lock must be
			their nick.  This allows people to have factoids of their own even
			without having special privileges.  To do this, they simply need
			to tell the bot ``lock foo'' where ``foo'' is the factoid key they
			wish to lock.  Unlocking adheres to the same set of rules except
			that you use ``unlock'' instead of ``lock''.

		\subsection{Replacing Factoids}
			Factoids can be replaced if the person requesting it has
			delete\_priv.  To do this, simply tell the bot ``no <factoid key>
			is <new factoid>'' and it will replace the factoid for that
			factoid key with your new factoid text.  We use delete\_priv for
			this because the mechanism behind it is implemented by simply
			deleting the factoid and recreating it.

		\subsection{Modifying Factoids}
			If a small portion of a factoid's value is fit to be changed, it
			can be done without replacing the entire factoid using simple
			regular expressions.  To do this, simply tell the bot ``<factoid
			value> =~ <set of regular expressions to apply, in order>''.  For
			example, if I set the factoid ``foo'' to have the value ``bar'', I
			can make it read ``blatz'' by telling the bot ``foo =~
			s/bar/blatz/''

		\subsection{Searching Factoids}
			You can search the factoids database in a number of ways.  You can
			search by factoid key, factoid value, or author (substrings of
			each).
			
			\subsubsection{Search by key}
				Obviously if you know what a factoid key is, you can simply
				request the factoid as described above.  If you want to see
				which factoid keys contain a certain substring, simply use the
				``listkeys'' function to do that.  If you want to see all keys
				with ``foo'' in the name, simply address the bot with
				``listkeys foo'' and it will list up to 15 distinct keys that
				have that substring in it, with a total count of how many do
				have that substring.

			\subsubsection{Search by value}
				If you know what a factoid contains, but aren't sure what the
				factoid key is, you can use the ``listvalues'' function to
				find the factoid you are looking for.  If you want to find the
				factoid which contains a certain word or phrase, simply
				address the bot with ``listvalues <phrase to search for>'' and
				it will display up to 15 distinct keys that contain that
				phrase as well as displaying a total count of keys that
				contain that phrase.

			\subsubsection{Search by author}
				You can see what factoids a certain user has created by
				searching for their name or nickmask using the ``listauth''
				command.  Simply address the bot with ``listauth <author name
				or substring>'', and it will display results as described in
				the previous two search sections.

		\subsection{Requesting Factoid Info}
			A variety of information about each factoid is stored, ranging
			from who authored to the factoid, to when it was created, to who
			locked it if anyone, to how many times it has been requested, etc.
			To get this information use the ``factinfo'' function.  Simply
			address the bot with ``factinfo <factoid key>'' and you will be
			greeted with a plethora of factoid information.
		
	\section{Larts, Punishes, and Praises}
		Lart is an abbreviation of the ``Luser Attitude Readjustment Tool'',
		an idea we got from the blootbot we used to use.  To use this against
		something or someone else (or even yourself), simply address the bot
		with the ``lart'' or ``punish'' command: ``(lart|punish) <whatever>''.
		You can optionally add a reason for the lart, by appending ``for
		<reason>''.  The action that the bot takes will describe the reason as
		well.  The actions (larts) that the bot dishes out are retrieved from
		a table in the bot's database, and chosen at random.

		Alternatively, you can do something nice for someone or something by
		using the ``praise'' tool.  The syntax is exactly the same as above
		except using the ``praise'' command instead of ``lart'' or ``punish''.

		\subsection{Adding Larts}
			If a user has the add\_lart\_priv, they can add a lart to the
			bot's database.  The syntax for doing so is to address the bot
			with ``add lart <lart text>''.  The lart text should reference the
			object of the lart using the text ``WHO'' for whatever is to be
			larted/punished.  For example, ``add lart pulls WHO's pants down''
			would result in someone getting larted having their pants pulled
			down by the bot.

			Praises are added in the same manner, but with ``add praise''
			instead of ``add lart'', although it still requires
			add\_lart\_priv.

	\section{Channel Statistics}
		These modules fit more into the ``fun'' category, and really 
		aren't particularly useful.  but they're fun.
		\subsection{Hehstats, et al}
			This module counts the number of times a user has said
			any of the special keywords it's set up to listen for.  The
			initial word it listened for was ``heh''.  The count is kept
			by a user's nickname, and stored in the database.  A word is
			only counted when it is the soul contents of a message to a
			channel.  To view the top 3 counts for a particular
			statistic, use the ``hehstats'' command, substituting for
			``heh'' the name of the statistic you want to check.

			If a user is abusing the stats, for instance by repeatedly
			saying ``heh'' to bring their stats up, a bot admin can use
			the ``hehreset'' command, in the form ``moobot: hehreset 
			Bradmont'' and that persons' stats of that type will be set
			to zero.  You can also use this for resetting someone's
			karma.
			
		\subsection{Quotes/Webstats}
			This module counts the total number of times a person has
			spoken in each channel, and store a random quote for that
			person.  The chances of a new quote replacing the old is
			ten in the total number of times that person has spoken in
			the channel, so the quotes change less and less often the
			more someone speaks.  To get the quote for a user in the
			current channel, use ``moobot:  quote Bradmont''.  This
			module was written to keep statistics to be put on a
			website -- we have CGIs that display the stats, but I'm not
			sure if/when we're going to release them.

	\section{'Net Lookup}
		These modules create connections to external servers 
		(mostly via HTTP, but not always) to look up information.
		\subsection{Google}
			This module searches google and returns a list of URLs
			(a maximum of 5) for the results.  Syntax is like this:
			``moobot:  google for a big cow''.
		\subsection{Slashdot}
			By saying to the bot ``moobot: slashdot'', the bot will
			download the current slashdot headlines, and print them to
			the channel.
		\subsection{Kernel}
			This module displays the latest versions of various Linux
			kernel releases.  It uses the output from the finger daemon
			running on kernel.org (try ``finger @kernel.org'' in your
			favourite unix shell).  Syntax:  ``moobot: kernel''.
		\subsection{Insults}
			This module generally makes fun of things.  It gets a random
			insult and then applies it to the target specified by the
			user.  An example:
			\begin{verbatim}
				<Bradmont> moobot: insult perl
				<moobot> perl is nothing but a dread-bolted gob of unintelligent slurpee-backwash.
			\end{verbatim}
		\subsection{Excuses}
			This will get you a random, tech related excuse.  Syntax:
			``moobot: excuse''.
		\subsection{Dict}
			This implements a simple client for the ``dict'' protocol.  It
			connects to dict.org and looks up the meaning of a given word.
			Syntax:  ``moobot: dict foo''.
		\subsection{Version}
			A quick lookup for the version of any software package that
			exists in the freshmeat database.  Example:
			\begin{verbatim}
				<Bradmont> moobot: version python
				<moobot> Latest version of python according to [fm]: 2.2.1c1
			\end{verbatim}
		\subsection{Dns Lookup}
			Uses the DNS server of the bot's host machine to resolve
			either a DNS  name into an ip address, or vice versa.
			Syntax:  ``moobot: nslookup sourceforge.net''.
		\subsection{Stock Quotes}
			This module returns the current price of a stock, by the
			company's symbol (I don't know what those things are supposed
			to be called...  I'm not into the whole stock market scene)
			Example:
			\begin{verbatim}
				<Bradmont> moobot: stockquote rhat
				<moobot> The current price of RHAT is 5.95
			\end{verbatim}
		\subsection{.deb Lookup}
			Queries packages.debian.org for information on package names
			passed in.  You can specify a branch if you like, but it's not
			required. Example:
			\begin{verbatim}
				<Strike> moobot: deblookup foo apt moobot
				<moobot> no package found for: foo (all) ;; apt 0.5.4 (stable)
				;; apt 0.5.4 (testing) ;; apt 0.5.4 (unstable) ;; moobot
				0.6.1-1 (testing) ;; moobot 0.6.1-1 (unstable) ;; 
			\end{verbatim}
	
	\subsection{Miscellaneous}
		\subsection{Run-Time Module Reloading}
			When debugging moobot modules, you often have to test them
			a number of times to fix all the errors.  This used to mean
			that every change you made required that the bot be restarted,
			which can take time, since its IRC connection has to be killed
			and restarted, the bot must be reinitialized, and so forth.

			However, MooBot now has the ability to reload its modules at
			run-time, so you won't have to restart it nearly so often.
			To reload a single module, use ``moobot: reload *modulename*''.
			The bot will reload just that one module.  To reload all modules,
			simply use ``moobot: reload''.  Only users with all\_priv may
			use these functions.

			MooBot also has the ability to load new modules at run-time.  Just
			plop them in the search path and load them using ``moobot: load
			<modulename>''.

	\section{Customizing/Writing Modules}
		See the Module Author Guide included with this distribution.

	\section{Bugs}
		Probably the easiest way to contact us with bug reports is to simply
		hop on IRC and join us in \texttt{\#moobot} on
		ltexttt{irc.ofrc.net} and if we are around, tell us what
		problems you are having and we might even test it right then and there.
\end{document}

