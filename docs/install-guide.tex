\documentclass{article}

\author{Daniel DiPaolo}
\title{Moobot Installation Guide}

\begin{document}

\maketitle
\newpage

\tableofcontents
\newpage

\section{Prerequisites}
Moobot is written in Python, and assumes an installation of at \textit{least}
Python 2.0 on the host machine, or greater.  Other modules are used as well,
but the ones we use that are not installed along with Moobot (as described in
this guide), are detected in the installation process, and advisement is given
accordingly.


\section{Obtaining Moobot}
There are a variety of methods in which one can obtain the sources for moobot.
You can either obtain one of our releases via some mirror or from our official
repository on sourceforge.net at
\texttt{http://www.sourceforge.net/projects/moobot}, where you can download
official releases, or use CVS to get the latest enhancements to Moobot.  The
only official way we support of obtaining moobot is, of course, through our
official repository at sourceforge.net, so that is the only method we will
describe here.
\subsection{Sourceforge.net official releases}
To obtain an official release of Moobot from sourceforge.net, simply go to the
project page at \texttt{http://www.sourceforge.net/projects/moobot}, and click
on the package you wish to download (moobot, in this case).  It should bring
you to a screen which lists all of the available releases of Moobot that we
have ever done.  Obviously the latest release is recommended, as it will have
fixed bugs in the previous versions, however you may feel free to pick
whatever version you like.  For now, the only style of packaging that we will
have will be zipped/tarred up source collections that you simply unzip and run
``as is'', however we do plan to offer other means of installation in the
future.

Once you have obtained the archive for the version of Moobot you wish to use,
extract it into an appropriate directory (any directory will work, just choose
one that makes sense within your filesystem).  The next step is installing
Moobot, which we cover in the next chapter.
\subsection{Using CVS}
To obtain the latest version of Moobot, even later than the most recent
release, you have to use CVS.  Thankfully, sourceforge.net provides us public
CVS space that can be accessed anonymously by anyone.  And here's how you use
it to get the latest Moobot source.

First, you will want to create a sort of ``sandbox'' in which you want to
install Moobot so as not to disrupt other system files.  You can even install
it in your own home directory if you wish (this is how I run it).  Once that
is done, change to that directory, and perform the two commands shown below
(note: the password is empty, simply hit Enter when prompted for a password):

\begin{verbatim}
cvs -d:pserver:anonymous@cvs.moobot.sf.net:/cvsroot/moobot login
cvs -z3 -d:pserver:anonymous@cvs.moobot.sf.net:/cvsroot/moobot co moobot
\end{verbatim}

This should create a \texttt{moobot} directory and place several files and
subdirectories in it.  Once this is completed, you may update to the most
recent version of the files you have by simply typing \texttt{cvs update} in
this directory.  To get any new files added to the project since then,
however, you may have to reperform the steps above.  Now you are ready to
install Moobot fully!

\section{Installing Moobot}
At this point in time, you should have some directory with Moobot sources in
it (as outlined in the previous section).  For the most part, you are actually
already done.  All that is left is to make sure certain Python modules are
installed and to set up database stuff if you wish.  To do this, we have
provided an installation script for you, \texttt{install.py}.  To run the
installation script, either make the install script executable using the
\texttt{chmod} command and then run it with \texttt{./install.py}, or simply
run it with \texttt{python install.py}.
\subsection{Checking for modules}
The first step in the installation is checking to see whether or not all the
Python modules that Moobot needs/can use are installed on your system.  It
should prompt you telling you what is going on and ask you to hit enter to
continue.  Once you do so, it should begin checking for one of a few modules
that may or may not exist on your system.  Here is a listing of what modules
Moobot is designed to use right now as well as an explanation of what purpose
they serve in the operation of Moobot.
\begin{itemize}
	\item[\texttt{gtk}]
		Along with \texttt{GTK}, provides a GUI for viewing and/or
		editing the contents of the database you choose to use.  This
		library consists of most of the actual GUI design function
		calls.
	\item[\texttt{GTK}]
		Along with \texttt{gtk}, provieds a GUI for viewing and/or
		editing the contents of the database you choose to use.  This
		library consists mostly of constants pertinent to the
		\texttt{gtk} module.
	\item[\texttt{irclib}]
		Provides the core IRC functionality required by Moobot, as
		well as by \texttt{irclib}, another required library.
	\item[\texttt{ircbot}]
		Contains the base class from which Moobot is derived, and
		provides very low-level bot functionality of Moobot
	\item[\texttt{MySQLdb}]
		Provides a Python interface for a MySQL database used for
		many things that Moobot can do
	\item[\texttt{PgSQL}]
		Provides a Python interface for a PostgresSQL database
		used for many things that Moobot can do
\end{itemize}
Some of these modules are considered \emph{crucial} to the usage of Moobot,
because without them, it simply won't work.  Others, are important, but not
detrimental to the operation of Moobot.  Lastly, some are merely cosmetic and
will not adversely affect the operation of Moobot at all.  The install script
explains the level of importance of each missing module, and you can make your
own decisions as to whether or not to install the missing modules before
proceeding with the installation.
\subsection{Configuring bot parameters}
The next step in the installation is to configure certain parameters that the
bot will use in running.  As stated in the install script, the values you
choose here are not final, so don't be afraid to enter in bogus entries as
long as you have the intent of fixing them later on.  The values will be
stored in the \texttt{moobot.conf} configuration file in the same directory as
the install script, and this file will be parsed for these parameters whenever
you run Moobot.  The questions for this are fairly straightforward, so we
won't cover them in this guide.

\emph{(NOTE: If you are using CVS and do not wish your \texttt{moobot.conf}
file to
be overwritten when you update to the latest version of CVS (because we do
include \texttt{moobot.conf} in the CVS tree), then simply copy your
\texttt{moobot.conf} file to \texttt{/etc/moobot.conf}.  Moobot checks this
file for values last, and these values will override any values found in a
\texttt{moobot.conf} file in the current directory.)}

\subsection{Choose a Database}
If you had one of the aforementioned database modules installed, you are now
prompted to choose which database you would like to use.  Once you select a
database type from the list of databases available to you, the installer will
try to create a symlink (symbolic link) from the database module that you
chose to \texttt{database.py}.  If your filesystem or operating system doesn't
support symbolic links, you must copy the file manually.

\subsection{Setting up the Database}
In order to use Moobot's database capabilities, a certain database structure
is expected to be in place, as well as at least one entry in a few tables.
This portion of the install sets that up.

The first thing you are asked is for the bot's username to be used with the
database.  The actual name chosen here isn't all that important, as long as
you don't try to use a name that is already existing in the database.
For example, ``moobot'' will suffice.

Next, you are asked for the host that is running the database.  For all
intents and purposes, this should almost always be ``localhost'' because in
99\% of cases, you will want the database to be local.  If you have reasons
for it to be otherwise, then you are probably smart enough to know what goes
here.

The name of the database is next and it, like the username, is rather trivial
as long as you don't use one that is existing already.  Again, ``moobot'' is
as good a choice as any for this option.

Then you are asked to enter in a password for the database (twice, for
verification purposes).  Use normal password-choosing conventions here and all
should be well.

It then asks if you are sure if you want to create all of this now.  Unless
for some reason you wish to put this off until later, you will want to say
``yes'' or ``y'' here.  It will then use the database admin tools on your
system to create a new database user, a new database, and to import the
database structure which can be seen in the appropriate \texttt{.sql} file
from our CVS.

\section{Finished!}
Congratulations!  You should now have a working Moobot.  Simply run
\texttt{python moobot.py} and it will connect to whatever servers you
specified in your config file.

\section{Bug Reports}
Probably the easiest way to contact us with bug reports is to simply hop on
IRC and join us in \texttt{\#moobot} on \texttt{irc.openprojects.net} and if we
are around, tell us what problems you are having and we might even test it
right then and there.
\end{document}

